%%
%% This is file `sample-sigconf.tex',
%% generated with the docstrip utility.
%%
%% The original source files were:
%%
%% samples.dtx  (with options: `sigconf')
%% 
%% IMPORTANT NOTICE:
%% 
%% For the copyright see the source file.
%% 
%% Any modified versions of this file must be renamed
%% with new filenames distinct from sample-sigconf.tex.
%% 
%% For distribution of the original source see the terms
%% for copying and modification in the file samples.dtx.
%% 
%% This generated file may be distributed as long as the
%% original source files, as listed above, are part of the
%% same distribution. (The sources need not necessarily be
%% in the same archive or directory.)
%%
%% The first command in your LaTeX source must be the \documentclass command.

\documentclass[sigconf]{acmart}
\usepackage{multirow}
%%
%% \BibTeX command to typeset BibTeX logo in the docs
\AtBeginDocument{%
  \providecommand\BibTeX{{%
    \normalfont B\kern-0.5em{\scshape i\kern-0.25em b}\kern-0.8em\TeX}}}

%% Rights management information.  This information is sent to you
%% when you complete the rights form.  These commands have SAMPLE
%% values in them; it is your responsibility as an author to replace
%% the commands and values with those provided to you when you
%% complete the rights form.
\setcopyright{acmcopyright}
\copyrightyear{2019}
\acmYear{2019}
\acmDOI{10.1145/1122445.1122456}

%% These commands are for a PROCEEDINGS abstract or paper.
\acmConference[Summer '19]{Summer '19: A Summer Research Internship on Database}{July 23--31, 2019}{Singapore, SG}
\acmBooktitle{Summer '19: A Summer Research Internship on Database,
  July 23--31, 2019, Singapore, SG}
\acmPrice{15.00}
\acmISBN{978-1-4503-9999-9/18/06}



\begin{document}

%%
%% The "title" command has an optional parameter,
%% allowing the author to define a "short title" to be used in page headers.
\title{Vision Trajectories: Scene, Method and Dataset}

%%
%% The "author" command and its associated commands are used to define
%% the authors and their affiliations.
%% Of note is the shared affiliation of the first two authors, and the
%% "authornote" and "authornotemark" commands
%% used to denote shared contribution to the research.



%%
%% By default, the full list of authors will be used in the page
%% headers. Often, this list is too long, and will overlap
%% other information printed in the page headers. This command allows
%% the author to define a more concise list
%% of authors' names for this purpose.
\author{Taige Hou}
\affiliation{%
	\institution{Peking University}
	\city{Beijing}
	\country{China} 
}
\email{houtiger@pku.edu.cn}

\author{Sheng Wang}
\affiliation{%
	\institution{New York University}
}
\email{swang@nyu.edu}

%%
%% The abstract is a short summary of the work to be presented in the
%% article.

\begin{abstract}
Nowadays, the widely existence of vision trajectories like handwriting \cite{VikramLR13}, sports player running \cite{DBLP:conf/kdd/WangLCJ19} and pedestrain moving \cite{GuptaJFSA18} requires efficient trajectory  database management. And in this paper, we will first introduce the application scenes of vision trajectory. Then we'll discuss the difficulty in the process of vision trajectory similarity searching, like trajectory normalization and dynamic time warping (DTW) computation complexity. After that, a brief description of available datasets would be given. 
\end{abstract}

%%
%% The code below is generated by the tool at http://dl.acm.org/ccs.cfm.
%% Please copy and paste the code instead of the example below.
%%
\begin{CCSXML}
	<ccs2012>
	<concept>
	<concept_id>10002951.10002952.10002953</concept_id>
	<concept_desc>Information systems~Database design and models</concept_desc>
	<concept_significance>500</concept_significance>
	</concept>
	</ccs2012>
\end{CCSXML}

\ccsdesc[500]{Information systems~Database design and models}


%%
%% Keywords. The author(s) should pick words that accurately describe
%% the work being presented. Separate the keywords with commas.
\keywords{database management, handwriting trajectories}

%% A "teaser" image appears between the author and affiliation
%% information and the body of the document, and typically spans the
%% page.


%%
%% This command processes the author and affiliation and title
%% information and builds the first part of the formatted document.
\maketitle

\section{Introduction}
Handwritten trajectories can also be applied to computer-human intereactions \cite{VikramLR13}. The paper proposes a novel way to control computers by moving fingers in the air. They demonstrate this human input approach through an example application of handwriting recognition. A 3D finger moving trajectory is captured by the sensor and then search for the most similar trajectory in the standard character trackings database. Because while capturing, there is no explicit gesture that indicates when a character starts or stops, so that each subtrajectory should be compared with all the standard trackings.  While the dynamic time warping (DTW)  based similarity search algorithm is so time-consuming  that searching in a 160-trajectories dataset may take about 10 seconds. So a more efficient way to manage these trajectory data is in need, especially when we are dealing with database that contains several millions of handwritten trackings. 






\section{Normalization}



\section{Dynamic Time Warping}
In measuring trajectory similairty, Dynamic Time Warping (DTW) is a well accepted method. Different from Euclidean Distance, DTW consider matching points within a certain time window, that's also how the name ``warping'' comes. But the expense of computation is very large, and several optimizations are proposed in \cite{RakthanmanonDTW, vldb/LB_keogh}, including lower bound pruning, early abandoning Z-Normalization, reordering early abandoning, reversing the Query/Document role in $LB_{keogh}$ and casacading lower bounds. Even though, the approximate computing complexity of DTW is still $O(mn)$, where $m$ and $n$ represent the length of query and candidate aequences separately. 

\section{Datasets}
There are datasets of vision trajectories of considerable size. For example, the CASIA database of Chinese character handwriting trajectories \cite{DBLP:conf/icdar/LiuYWW11} consists of over 160 million trajectories of more than 3000 classes. And trajectory databases of other languages are also available, like English \cite{UJIPen}, Japanese \cite{icdar/Japanese}, Vietnamese \cite{Vietnamese}, etc. In \textbf{Table 1}, we show the statistics of CASIA database.


\begin{table}
	\caption{Statistics of CASIA Database}
	\label{tab:CASIA}
	\begin{tabular}{|l|l|l|l|l|}
		\hline
		\multirow{2}{*}{dataset} & \multirow{2}*{writers} & \multicolumn{3}{|c|}{character samples} \\
		\cline{3-5}
		& & total & symbol & Chinese/class \\ 
		\hline
		OLHWDB1.0 & 420 & 1,694,741 & 71,806 & 1,622,935/3,866 \\
		\hline
		OLHWDB1.1 & 300 & 1,174,364 & 51,232 & 1,123,132/3,755 \\
		\hline
		OLHWDB1.2 & 300 & 1,042,912 & 51,181 & 991,731/3,319 \\
		\hline
		total & 1,020 & 3,912,017 & 174,219 & 3,737,798/7,185\\
		\hline
	
	\end{tabular}
\end{table}

%%
%% The next two lines define the bibliography style to be used, and
%% the bibliography file.
\bibliographystyle{ACM-Reference-Format}
\bibliography{sample-base}






\end{document}
\endinput
%%
%% End of file `sample-sigconf.tex'.
